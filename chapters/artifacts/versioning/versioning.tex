%! TEX root = ../../../main.tex

\section{Versioning}%
\label{sec:Versioning}

\subsection{The Need for Good Versioning Conventions}%
\label{sub:The_Need_for_Good_Versioning_Conventions}
Unlike in the monolithic architecture, a continuously deployed microservice
architecture is comprised of many components that each have their own version.
Hence it has to be decided case by case which versioning scheme best fits each
microservice. Further, as many microservices provide some kind of \ac{API}
(mostly \ac{REST}) some form of versioning has to be used. Without versioning,
the application implementing the \ac{API} would always consume the latest
version of the microservice. This implies that any non-backward-compatible
update to the microservice would break all client implementations of the
microservice's \ac{API}. Not only is the client able to pin its implementation
to a specific \ac{API} version. The client is further able to select any
version of a resource that best fits its implementation. In addition, a good
versioning scheme allows developers to quickly assess whether a new version
breaks the service's backward-compatibility. Lastly, microservice do not only
provide services to other microservices. E.g.\ a \textit{frontend} microservice
directly serves the end user. Hence the versioning scheme used in such a
microservice also serves a communicative purpose.

\subsection{Versioning in a Microservice Environment}%
\label{sub:Versioning_in_a_Microservice_Environment}
Chapter~\ref{sub:Versioning} already introduced the two main versioning schemes
\textit{SemVer} and \textit{CalVer}. As microservices mostly provide their
\ac{API} through the \ac{REST} paradigm, the chapter also presented the three
main possibilities to version a \ac{REST} \ac{API}.

\subsection{Evaluating XXX Versioning}%
\label{sub:Evaluating_XXX_Versioning}

\subsection{Advancing Continuous Deployment of Microservices using XXX Versioning}%
\label{sub:Advancing_Continuous_Deployment_of_Microservices_using_XXX_Versioning}

\subsection{Economic benefits of XXX Versioning}%
\label{sub:Economic_benefits_of_XXX_Versioning}

