%! TEX root = ../../main.tex

\subsection{Kubernetes}%
\label{sub:Kubernetes}

The practise of running microservices in a production environment is not only
used by big technology companies. The United Kingdom government's Department
for Work and Pensions states to be running one of the largest microservice
architectures in Europe. The country's public services \textit{universal
credit} system fully runs as a microservice architecture
\autocite{LoweLeadingwaymicroservices2016}. Yet how is it possible to
orchestrate such a huge amount of microservices?

Kubernetes is a container orchestration solution that is capable of
automatically deploying, scaling and managing containers
\autocite{AuthorsProductionGradeContainer}. It was introduced in 2014 by Google
and is based on the experience Google gained while developing among other
things \textit{Borg} and \textit{Omega}. Borg and Omega were applications
internally developed by Google to manage their thousands of applications and
services \autocite{LuksaKubernetesAction2017}.

This chapter will shortly introduce the main concepts of Kubernetes and give an
overview of the Kubernetes components that are relevant to the research
questions. If needed, The official Kubernetes documentation
\autocite{AuthorsProductionGradeContainer}, which covers all system components
in detail, can be used as a supplement to this thesis.

