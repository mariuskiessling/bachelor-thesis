%! TEX root = ../../main.tex

\subsection{Document Structure}%
\label{sub:Document Structure}

In the upcoming chapter~\ref{sub:Design_Science_Research}, the methodology that
is utilised to produce solutions for these problems is presented.

Once the organizational aspects are established, chapter~\ref{sec:Background}
provides the grounding for the solution search process. It covers all the
relevant theoretical topics from microservices over the container orchestration
solution \textit{Kubernetes}, the continuous build and deployment process of
microservices, the different environments in which a continuous deployment can
be performed to the versioning schemes commonly used in software development.

Chapters~\ref{sec:Versioning} and~\ref{sec:Handling_Kubernetes_Manifests} then
try to produce solutions to the problems stated in the introductory chapter.
\todo{Add listing of problems that are discussed in ch. 3-5}. \acf{DSR} defines
clear rules on how these solutions should be produces. Part of these rules is
to discuss and test them using some form of validation. These chapters will
perform \acs{DSR}'s full research cycle including the development and
discussion of solutions.

Chapter~\ref{sec:Conclusion} summarizes the finding of this thesis. Finally,
chapter~\ref{sec:Outlook} provides an outlook on what further research can be
conducted on the topic of continuously deployed microservices. It also states
how the research conducted as part of this thesis can be improved or
supplemented.
