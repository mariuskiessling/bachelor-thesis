%! TEX root = ../../main.tex

\subsection{Project Scope}%
\label{sub:Project_Scope}

The research made in this thesis is performed in the environment of the
\textit{dsP-IT} project. The project is hosted by a group of dentists who
founded \textit{dentalsynoptics}, a registered non-profit association.
Dentalsynoptics has made it its mission to standardize the processes and tasks
commonly performed in a dental office. The project's goal is to transform the
analogue handbook of processes developed by dentalsynoptics using a modern
microservice architecture into digital processes and workflows
\autocite{HomepageDentalSynoptics2018}.

The exploration of the theoretical foundation of this thesis does not reference
the dsP-IT project directly. This way a project-independent review of literature
and discovery of theoretical groundwork can be guaranteed.

In addition, chapters~\ref{sec:Versioning}
and~\ref{sec:Handling_Kubernetes_Manifests} do not directly refer to components
of the dsP-IT project. The development of artefacts rather takes place on an
abstract layer above the project's actual components. This way, the project's
full suite of components does not have to be introduced. Only a selected number
of components which are needed to solve the problem at hand are presented
conceptionally. Furthermore, it is ensured that all research can be easily
transferred to applications experiencing similar issues as the ones examined in
this thesis. Nonetheless, \ac{DSR} artefacts can both be developed and tested
in the dsP-IT's practical environment. Thus, the research is not bound but
rather guided by a practical project.
