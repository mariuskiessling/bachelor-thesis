%! TEX root = ../../main.tex

\subsection{Design Science Research}%
\label{sec:Design_Science_Research}

When designing the optimal information system that is able to continuously
deploy microservices, the obvious outcome is a practical model. This model is
embedded in some technical environment but may not contribute any scientific
findings. That is the reason for why this thesis uses the \ac{DSR} approach for
designing information systems. \ac{DSR} is an conceptual framework that can be
applied to any research project. It defines seven principal guidelines that
govern the way research should be conducted. This section will explore the
workings of \ac{DSR} and how it will be utilised in this research endeavor.

In general, the goal behind \ac{DSR} is to design \ac{IT} \textit{artefacts}.
An artefact can not only be an instantiation, e.g.\ a software prototype, but
can also be a construct, model and method that is utilised in the development
and usage of information systems \autocite[p.
82]{VonAlanDesignscienceinformation2004}.

The creation of artefacts is regulated by seven guidelines. These guidelines
assure that the requirements for the conducted research are apprehensible for
both the researcher as well as the reader \autocite[p.
82]{VonAlanDesignscienceinformation2004}.

\LTXtable{\textwidth}{tables/design_science.tex}

In most cases, an artifact does not represent a complete information system. It
much more tries to capture the ideas, methods and processes that are needed to
design and use an information system \autocite[p.
83]{VonAlanDesignscienceinformation2004}.

% TODO: Beziehung der obigen Aussage auf das Projekt
% TODO: Beschreiben des Vorgehens mit Design Science Research in diesem Projekt
