%! TEX root = ../../main.tex

\subsection{Microservices}%
\label{sub:Microservices}

There exist many different definitions of the term \textit{microservice}.
\autocite{DragoniMicroservicesyesterdaytoday2016} defines a microservice to be
a \enquote{cohesive, independent process interacting via messages}. The
definition of \autocite{MikeAmundsenMicroserviceArchitecture2016} also takes
the architectural aspect into account stating that a \textit{microservice
architecture} is \enquote{style of engineering highly automated, evolvable
software systems made up of capability-aligned microservices}. Regardless of
the specific definition, one common concept becomes clear: A microservice
should be a small independent piece of software that communicates using
messages and can be deployed autonomously.

This chapter will cover the origins of the microservice architecture, introduce
a practical example of a microservice and outline the problems that need to be
addressed when deploying a microservice.

\subsubsection{History}%
\label{ssub:History}

When designing an application, one option is to use a monolithic architecture.
A monolithic application is developed by one big group of developers and the
source code is maintained inside a single repository. The monolithic
application offers a number of functionalities (also called \textit{services})
that can be consumed. A team of operators is responsible to deploy and manage
the application so that it can be consumed \autocite[p.
584]{VillamizarEvaluatingmonolithicmicroservice2015}. The services of a
monolith can not be executed independently from one another. Hence, a monolith
exists as a single executable artefact \autocite[p.
1]{DragoniMicroservicesyesterdaytoday2016}.

Monolithic applications are hard to deploy in a distributed system without
introducing some kind of middleware that handles the way the monolith is
distributed. In addition to this, monolithic application suffer additional
drawbacks. Maintaining a monolith can pose a big problem due to its huge code
base and many internal and external dependencies. Furthermore, deploying a new
version can cause major downtimes since the whole application has to be
(re)started. From a development standpoint, monoliths also pose the problem of
restricting the technology stack once it has been defined. This can produce an
environment that is now longer suitable for all new services added to the
monolith. Lastly scaling a monolith is not possible indefinitely \autocite[p.
2]{DragoniMicroservicesyesterdaytoday2016}. When scaling an application
\textit{vertically}, resources (e.g.\ RAM) are added to the machine executing
the application. Contrary to this approach, when scaling \textit{horizontally}
the application gets distributed onto a number of machines. This however
requires the use of a distributed middleware or code changes \autocite[Ch.
1.1.1]{LuksaKubernetesAction2017}.

The \ac{SOA} solves a lot of the problems afflicting the monolithic
architecture. Rather than providing all services as part of one application, in
\ac{SOA} the application is split up into multiple small \textit{business
applications} that each offer only a small hand of services. Each business
application is maintained by a specialised development team. The business
applications offer their services to other services or directly to consumers
through a set of protocols; the primarily used protocol is \ac{SOAP}. The
operation of all business applications is handled by a separate operations
teams \autocite[p.  584]{VillamizarEvaluatingmonolithicmicroservice2015}.

Even though the \ac{SOA} approach does solve a lot of problems the monolithic
architecture suffers from, several problems remain. Implementing the \ac{SOA}
into existing and new applications can be difficult resulting in hight costs.
Additionally the technology that is needed to route the requests to the correct
business application is not designed to run inside a modern cloud environment
\autocite[p. 584]{VillamizarEvaluatingmonolithicmicroservice2015}.

The goal of the microservice architecture is to adopt the advantages of the
\ac{SOA} while solving the problems the monolith architecture suffers from
\autocite[p. 584]{VillamizarEvaluatingmonolithicmicroservice2015}.

\autocite{VillamizarEvaluatingmonolithicmicroservice2015} proposes the
questions whether the microservice architecture is a new architectural style or
simply another term for the already existing \ac{SOA}.
\autocite{VillamizarEvaluatingmonolithicmicroservice2015} concludes that the
microservice architecture can be viewed as a subset of the \ac{SOA}. Besides
that it additionally focuses on greater agility. Altogether,
\autocite{DragoniMicroservicesyesterdaytoday2016} defines the microservice
architecture to be a distributed application which \textit{modules} are only
microservices. The term \enquote{module} is synonymous with the term
\enquote{service} used in the description of the monolithic architecture and
\ac{SOA}.

In summary it can be said that the microservice architecture constitutes the
natural evolution of software architectural patterns.

\subsubsection{A Practical Example}%
\label{ssub:A_Practical_Example}

\subsubsection{Deployment - A Software-based View}%
\label{ssub:Deployment_-_A_Software-based_View}

\subsubsection{Problems}%
\label{ssub:Problems}


