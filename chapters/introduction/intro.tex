%! TEX root = ../../main.tex

The internet is filled with users. In 2016 the estimated amount of active
internet users was 3,26 billion. It is predicted that in 2021 this number will
rise to 4,14 billion. The results are a growth of about one billion new monthly
active users in only five years
\autocite{eMarketerAnzahlderInternetnutzer2017}. As reported by Cisco Systems,
one of the leading network hardware manufacturers in the world, the monthly
internet traffic already totalled 122 exabytes per month in 2017. Over the
course of five years, it is predicted that the volume of traffic on the
internet will rise to staggering 396 exabytes in 2022
\autocite{SystemsDatenvolumendesglobalen2018}. To put this into perspective,
one exabyte is equal to one million terabytes.

The problem however is that all of this traffic is not distributed evenly
across the day. A news website might e.g.\ be consumed by more users in the
morning or after work instead of during the day. This results in a high number
of requests and thus a high load on the website's servers during the peak load
times. In order to keep up with the demand the servers serving the responses
have to be equipped accordingly. The result is a highly powerful computing
cluster that is able to serve users during peak load times but idles the rest
of the day. Not only is such behaviour wasting energy resources, it costs the
company operating the site a lot of money. One possibility to solve this issue
is to dynamically scale the infrastructure in accordance with the requests.
During a time of high load, more servers are deployed to keep up with user
requests. Whenever the load decreases, the number of servers is scaled down.

To ensure this functionality an important precondition has to be fulfilled: An
application's infrastructure has to be as best as possible decoupled. In this
context the term infrastructure does not denote the physical layer (hardware,
network, etc.) but the virtual components an application is made up of. This
includes both supporting software like databases and caching as well as the
application's actual feature set. The feature set of web shopping application
might e.g.\ be comprised of managing a user's shopping basket, listing a
product's properties, a search engine that finds products suitable for a
user and a help centre with frequently asked questions. Only if all these
components of an application are decoupled, they can be scaled individually.
Revisiting the web shop example, a high load on the shops search engine does
not require that the shopping basket and help centre are scaled up as well.
Decoupling allows application operators to target specific feature sets and the
corresponding supporting software when scaling.

Such flexibility needs a major change in thinking about a software's
architecture. Ultimately the \textit{monolithic} architecture pattern has to be
dropped in favour of \textit{microservices}. Inspired by the services oriented
architecture, the \textit{microservice architecture} favours splitting the
concerns of an application into many small parts called microservices. The
benefits of the microservice architecture are fully explored further along in
this thesis. Not only small organisations and companies jumped on the
microservice bandwagon. The British government
\autocite{LoweLeadingwaymicroservices2016} and major online giants like Amazon
\autocite{JenkinsVelocityCulture2011} deploy the power and flexibility of
microservices in their production environments.

With the introduction of microservices, a number of organisational changes are
associated. The shift towards \textit{DevOps} gives developers more power about
their application's whole lifecycle, including its operation. Instead of being
depended on additional teams, developers are free to deploy an application
whenever they consider it to be stable.

\acf{CI} and \acf{CD} further support the DevOps paradigm by allowing to
continuously build and deploy an application. Although not all applications are
built and deployed equally. When deploying microservices instead of monoliths,
a number of things have to be considered. This thesis asses \todo{Update number
at the end of thesis} three problems that are associated with continuously
deployed microservice architectures. The goal is to find solutions, e.g.\ in
the form of models, to these problems. To get started,
chapter~\ref{sub:Problem_Statement} will introduce the three problems in question.
