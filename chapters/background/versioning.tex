%! TEX root = ../../main.tex

\subsection{Versioning}%
\label{sub:Versioning}
The topic of versioning does not only come up in the context of microservices.
In traditional software releases, e.g.\ of \textit{Microsoft Windows}, a number
of different versioning schemes were used; from Version 1.0, 3.1, 95, 2000, XP
and 8 up to 10. The number schemes changed without retaining a consistency. For
programmers, a consistent versioning scheme is necessary in order to properly
deal with dependencies \autocite{Preston-WernerSemanticVersioning2.0.0}. Thus
this chapter will introduce two schemes software can be consistently versioned
with. It will further introduce an approach of versioning \acp{API}.

\subsubsection{SemVer}%
\label{ssub:SemVer}
The \textit{Sem}antic \textit{Ver}sioning scheme \textit{SumVer}
\autocite{Preston-WernerSemanticVersioning2.0.0}, authored by Tom
Preston-Werner, specifies eleven rules for how a version string should be
structured. It further describes when which part of the version string may be
incremented. Each version number consists of three parts:

\begin{itemize}
  \item \texttt{MAJOR}
  \item \texttt{MINOR}
  \item \texttt{PATCH} 
\end{itemize}

Together they form a version number in the format of
\texttt{MAJOR.MINOR.PATCH}. The \texttt{PATCH} part must only be incremented
whenever bug fixes are introduced that do not break the software's
backward-compatibility. The \texttt{MINOR} part has to be incremented each
time a release includes new features while still being backward-compatible.
Lastly, an incremented \texttt{MAJOR} part indicates that the release breaks
the backward-compatibility of the software's \ac{API}. In addition to these
three main parts, SemVer also includes provisions for how to deal with
pre-released software, e.g.\ an alpha version, and how to attach metadata to
the version number. A pre-release version must be added as a postfix,
separated by the \texttt{-} character. The postfix may only include
alphanumeric characters (including \texttt{-}) and may consist of multiple
parts separated by the \texttt{.} character. It may optionally end with a patch
number of said pre-release. The patch number is separated by the \texttt{.}
character. All in all a pre-released software version number may look like
this: \texttt{2.0.0-alpha.public.4}. Any metadata is added at the end of the
version number separated by the \texttt{+} character. The character
requirements for the metadata part are identical to the ones of pre-release
flag. The metadata also may consist of multiple parts that are separated by the
\texttt{.} character. The metadata part is intended to include information
about the build of a software. A full version number, according to SemVer, for
an application that is currently in beta phase and is built on the first of
September could look like this: \texttt{2.0.0-beta.12+2019-09-01}.

These guidelines provide developers with an easy way to identify whether or not
a dependency of their project may break its backward-compatibility when
updating it to a newer version. This however is only possible if the dependency
fully adheres to the SemVer specifications. In a study
\autocite{RaemaekersSemanticVersioningversus2014} of 22,205 libraries published
to the \textit{Maven Central Repository}, a place where Java developers can
share programming artifacts, researchers examine whether version identifiers
actually correspond with breaking changes. They conclude that breaking changes
are introduces both in one third of major releases as well as in one third of
minor releases. Hence they argue that the use of semantic versioning in a
project is no actual guarantee that said project follows the SemVer
specification. \ac{API} backward-compatibility can not be determined by looking
at a project's version number. This assessment disproves the main purpose the
SemVer specification was created for.
