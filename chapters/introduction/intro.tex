%! TEX root = ../../main.tex

The internet is filled with users. In 2016 the estimated amount of active
internet users was 3,26 billion. It is predicted that in 2021 this number will
rise to 4,14 billion. The results show a growth of about one billion new
monthly active users in only five years
\autocite{eMarketerAnzahlderInternetnutzer2017}. As reported by Cisco Systems,
one of the leading network hardware manufacturers in the world, the monthly
internet traffic already totalled 122 exabytes per month in 2017. Over the
course of five years, it is predicted that the volume of traffic on the
internet will rise to staggering 396 exabytes in 2022
\autocite{SystemsDatenvolumendesglobalen2018}. To put this into perspective,
one exabyte is equal to one million terabytes. These enormous numbers introduce
the question how it is possible to deal with them.

The problem is that all of this traffic is not distributed evenly across the
day. A news website e.g.\ might be consumed by more users in the morning or
after work instead of during the day. This results in a high number of requests
and thus a high load on the website's servers during the peak load times. In
order to keep up with the demand the servers serving the responses have to be
equipped accordingly. The result is a highly powerful computing cluster that is
able to serve users during peak load times but idles the rest of the day. Not
only is such behaviour wasting energy resources, it costs the company operating
the site a lot of money. One possibility to solve this issue is to dynamically
scale the infrastructure in accordance with the requests.  During a time of
high load, more servers are deployed to keep up with user requests. Whenever
the load decreases, the number of servers is scaled down.

To ensure this functionality, an important precondition has to be fulfilled: An
application's infrastructure has to be decoupled as best as possible. In this
context the term infrastructure does not denote the physical layer (hardware,
network, etc.) but the virtual components of which an application is made up.
This includes both supporting software like databases and caching as well as
the application's actual feature set. The feature set of a web shopping
application might e.g.\ be comprised of managing a user's shopping basket,
listing a product's properties, of a search engine that finds products suitable
for a user and of a help centre with frequently asked questions. Only if all
these components of an application are decoupled, can they be scaled
individually. Revisiting the web shop example, a high load on the shop's search
engine does not require that the shopping basket and help centre be scaled up
as well. Decoupling allows application operators to target specific feature
sets and the corresponding supporting software when scaling. One approach to
decoupling a software is called the \textit{microservice architecture}.

Such flexibility needs a major change in thinking about a software's
architecture. However not only the software's architecture has to adapt. A
number of organisational changes in favour of a \textit{DevOps} culture are on
the daily agenda of a modern software company. Not only major online giants
like Amazon \autocite{JenkinsVelocityCulture2011} but also the British
government \autocite{LoweLeadingwaymicroservices2016} realised this. The shift
towards \textit{DevOps} gives developers more power about
their application's whole lifecycle, including its operation. 

\textit{\acf{CI}} and \textit{\acf{CD}} further support the DevOps paradigm by
allowing to continuously build and deploy an application. However not all
applications are built and deployed equally. When deploying microservices, a
number of things have to be considered. This thesis assess two problems that
are associated with continuously deployed microservice architectures. The goal
is to find solutions, e.g.\ in the form of models, to these problems. To get
started, chapter~\ref{sub:Problem_Statement} will introduce the two problems
in question.
