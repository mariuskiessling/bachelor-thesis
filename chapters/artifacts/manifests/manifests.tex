%! TEX root = ../../../main.tex

\section{Handling Manifests}%
\label{sec:Handling_Manifests}

\subsection{A Call for Better Manifest Management}%
\label{sub:A_Call_for_Better_Manifest_Management}
When configuring a microservice architecture to be deployed to a Kubernetes
Cluster, a lot of configuration files accrue. Each of these manifest files has
to be managed in some form. However, Kubernetes does not enforce any structure
upon how these files are stored and continuously managed. This lack of
standards makes it easy to get started with Kubernetes but can leave developers
stranded when working inside complex microservice architectures. In addition,
the microservice chapter (\ref{sub:Microservices}) already stated that
microservices are often developed in specialised teams. Thus, without any
management standard enforced by Kubernetes each microservice team can
theoretically define its own practises best fitting their preferences. This
might be tolerable inside a microservice team. However with respect to the
overall microservice architecture, different standards for managing Kubernetes
manifests makes it harder for developers to understand the deployment structure
of their colleagues' microservices. Furthermore, with a common manifest
management standard every microservice can be set up the same way to be
deployed using a \ac{CI}/\ac{CD} system. Lastly, when needing support from an
external infrastructure operations team, a common manifest standard helps all
involved parties to exchange information faster and with less friction.
