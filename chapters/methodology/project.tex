%! TEX root = ../../main.tex

\subsection{Project Scope}%
\label{sub:Project_Scope}

The research made in this thesis is performed in the environment of the
\textit{dspIT} project. The project is hosted by \textit{dentalsynoptics}, a
group of dentists. The project's goal is to digitize the analog handbook of
processes developed by dentalsynoptics using a modern microservice
\autocite{HomepageDentalSynoptics2018} architecture.

The exploration of the theoretical foundation of this thesis does not reference
the dspIT project directly. This way a project-independent review of literature
and discovery of theoretical groundwork can be guaranteed.

In addition, chapters~\ref{sec:Versioning}
and~\ref{sec:Handling_Kubernetes_Manifests} do not refer directly to components
of the dspIT project. The development of artifacts rather takes place on an
abstract layer above the project's actual components. This way, the project's
full suite of components does not have to be introduced. Only a select number
of components which are needed to solve the problem at hand are presented
conceptionally. Furthermore, it is ensured that all research can be easily
transferred to applications experiencing similar issues as the ones examined in
this thesis. Nontheless, \ac{DSR} artefacts can both be developed and tested in
the dspIT's practical environment. Thus, the research is not bound but rather
guided by a practical project.
