%! TEX root = ../../main.tex

\subsection{Problem Statement}%
\label{sub:Problem_Statement}

Kane and Matthias suggest that \enquote{shipping software at the speed expected
in today's world is hard to do well} \autocite[p.
2]{SeanPKaneDocker&Running2018}. When implemented well, microservices can
contribute towards faster shipping software. However not only the speed at
which software is shipped these days is difficult. With the introduction of
microservices multiple new challenges arise such as how these microservices can
be monitored, scaled, optimized and orchestrated \autocite[p.
67]{TrihinasDevOpsasService2018}. Besides that, a number of security challenges
emerge when using the microservices paradigm
\autocite{YaryginaOvercomingSecurityChallenges2018}.

This thesis will mainly focus on the orchestration challenge with only one
exception. It will try to answer the question \textit{how can microservices be
continuously developed and deployed}. This question however is far to extensive
in order to be discussed and answered as a whole. Thus, this thesis will focus
on three \todo{Update number accordingly when finishing the thesis.} \textit{problem domains} each with their own encapsulated research
question.

\label{link:problem_domains}
\begin{itemize}
  \item How should a continuously deployed microservice be versioned?
  \item How should microservice manifests be managed and deployed?
  \item How can the different deployment stage environments be defined?
\end{itemize}

The answers to these questions combined with the underlying background
information will give an overview on the landscape of microservices and how
they can be continuously developed and deployed.

In order to answer the question of each problem domain, the \ac{DSR} framework
will be used. Chapter~\ref{sub:Design_Science_Research} outlines the workings
of \ac{DSR} and how it is applied to each problem domain.
