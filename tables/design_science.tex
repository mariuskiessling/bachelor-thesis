\begin{longtable}{lXXX}
\label{tab:design_science_guidelines}
\vspace{0.25cm}\\
\toprule
\textbf{Guideline} & \textbf{Description} \\ 
\midrule
\endfirsthead
\toprule
\textbf{Guideline} & \textbf{Description} \\ 
\midrule
\endhead
\midrule
& \hspace*{\fill} \small{Table continues on next page.}
\endfoot
\bottomrule
\caption[The seven research principles defined by the \acs{DSR} framework.]{The seven research principles defined by the \ac{DSR} framework (adapted from \autocite[p. 84]{VonAlanDesignscienceinformation2004}).}
\endlastfoot
\textit{Guideline 1:} Design as an Artefact & One outcome of \ac{DSR} is an artefact (either a construct, model, method or an instantiation). \\
\textit{Guideline 2:} Problem Relevance & The technology-based solutions that are produced solve actual business problems. \\
\textit{Guideline 3:} Design Evaluation & Evaluation methods must be used to show the utility, quality and efficacy of an artefact. \\
\textit{Guideline 4:} Research Contributions & \ac{DSR} does not only produce practical results but also theoretical findings regarding design artefact, foundations or methodologies. \\
\textit{Guideline 5:} Research Rigour & The construction and evaluation of artefacts has to be done using rigorous methods. \\
\textit{Guideline 6:} Design as a Search Process & The design process is built upon an iterative process that employs all necessary means. The laws governing the problem environment must be adhered to.\\
\textit{Guideline 7:} Communication of Research & The research has to be presented both to technical as well as business oriented audiences. \\
\end{longtable}
