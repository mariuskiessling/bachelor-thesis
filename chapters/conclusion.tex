%! TEX root = ../../main.tex

\section{Conclusion \& Outlook}%
\label{sec:Conclusion}

The shift from the monolithic application architecture to the microservice
architecture solves a lot of problems. E.g.\ different architecture components
can be scaled individually to better deal with requests that are unevenly
distributed across the day. However, with the introduction of microservices a
number of new problems are introduced. This thesis confined its research to
problems are are associated with the deployment of microservice architectures.
Since this is a broad field, two particular problems were selected.

\begin{itemize}
  \item How should a continuously deployed microservice be versioned?
  \item How should microservice manifests be managed and continuously deployed?
\end{itemize}

These two research questions were selected out of a practical necessity. Like a
monolithic piece of software, a microservice architecture has to be versioned
somehow. Though due to the many parts that make up a microservice based
architecture, no apparent versioning scheme could be determined. In addition,
the deployment of microservices can be admirably well implemented automated.
This lays in the nature of the microservice architecture as its different
components can be interchanged isolatedly. Yet, a number of new configurations,
Kubernetes manifests, are often affiliated with the deployment of a
microservice architecture. Each service needs specific deployment
configurations. Furthermore, a number of deployment configurations exists which
cannot be assigned to a specific microservice. This calls for a concept which
states how the number of deployment configurations should be managed.

The actual task of deploying a microservice architecture was considered to be
trivial. Therefore, it was only outlined as part of the background chapter and
not further examined.

To solve the two problems outlined above, the \ac{DSR} research framework was
used to guarantee an outcome that would be tested in practise as well as
theoretically applicable to similar problems. Hence, the workings of \ac{DSR}
were summarized in chapter~\ref{sub:Design_Science_Research}. Next to the
methodology, this thesis also provided the necessary background knowledge on the
topics that were later utilized in the development of solutions in
chapter~\ref{sec:Background}. This thesis therefore also acts as an information
piece on architectural changes from monoliths to microservices, over the
container orchestration solution Kubernetes, the Helm package manager for
Kubernetes, \ac{CI} and \ac{CD} systems in general and with regard to
microservices up to different versioning schemes.

The first research question was answered in chapter~\ref{sec:Versioning}. A
classification model for microservices was introduced which distinguishes
between \textit{consuming} and \textit{producing} microservices. Based on this
classification, the terms \textit{manual} and \textit{continuous} versioning
were coined. If applicable, microservices should be assigned a version number
using a continuous versioning scheme. One example for a continuous versioning
scheme is CalVer. CalVer defines a set of rules for version numbers that are
based on calender dates. Thus, they can be assigned automatically without human
intervention. In cases, where the version number of a microservice has to
indicate a break in backward-compatibility, a manual versioning scheme like
SemVer should be used. If SemVer is technically required, it can also be
customized to be partially continuous by e.g.\ appending a rolling build
version at the end of the version string. In addition, a guideline was
developed for developers which helps them to identify the best versioning
scheme for their current project.

The second research question was resolved in
chapter~\ref{sec:Handling_Kubernetes_Manifests}. First, a naive deployment
approach was outlined. On the basis of this approach, a new one was developed
which solves a number of issues related to deployment configuration management.
The newly developed \ac{CIST} deployment concept is based on Helm, a package
manager for Kubernetes. With \ac{CIST}, a microservice architecture's
deployment configuration state is defined a central location, a Helm chart
repository. The architecture is divided into \textit{component charts} and
\textit{master charts}. A component chart defines how a single microservice has
to be deployed. A master chart glues all microservices of an microservice
architecture together and defines the architecture's underlying infrastructure.
From the chart repository, a complete microservice architecture including all
its components can be deployed without any additional manual interaction. The
only component that has to be deployed is the master chart. Everything else is
put together by Helm. \ac{CIST} also defines the procedure that produces said
central state. This procedure can be applied to any microservice architecture
which developers aim to simplify the project's deployment process. In addition
to an easier deployment process, Helm offers additional advantages like a
simple rollback strategy.

The outcome of both research questions can be easily transferred to similar
projects. However, the practical implementations of both solutions only
represent a reference and no completed solutions. Thus, additional research can
be performed on developing an implementation which is platform-independent as
the implementation proposals in this thesis were always based on Microsoft
Azure DevOps. Additionally, as already pointed out, additional studies could
further examine the theoretical concepts' economic benefits. Furthermore, the
applicability of the \ac{CIST} deployment model to other architectural styles
can be studied. Nonetheless, all theoretical findings of this thesis are
applicable without constraints in a project that uses a microservice
architecture as they do not require any technological stacks except for
Kubernetes and Helm.

Finally, this thesis' findings support developers whose applications are
publicly accessible on the internet to continuously serve their applications to
the demand of billions of possible users.
