% !TEX root =  main.tex

A software can only be utilized if it is made available to its users. The means
with which this process is executed depends, among other things, on an
application's architecture. With the shift from the monolithic to the
microservice architecture and the introduction of the DevOps paradigm, a
tailored deployment concept is vital. In addition, a range of new technology
stacks like Kubernetes are needed to deal with the requirements imposed by the
microservice architecture.

This thesis examines two problems which are related to a continuously deployed
microservice architecture. It focuses on how such an architecture should be
versioned and by what methods the deployment configuration, in this case
Kubernetes manifests, should be managed.

The Design Science Research methodology framework is individually applied to
each of the two research questions. The framework guarantees that the derived
solutions are both tested in practise as well as theoretically transferable.

Next to a classification scheme for microservices a \textit{continuous
versioning} concept is developed which helps developers to select the right
versioning scheme for their microservice architecture. In addition, the
\textit{Centralized Infrastructure State} deployment concept is introduced.
The concept defines how a microservice architecture's deployment configurations
should be managed and how they can be continuously deployed.

These concepts help developers to better manage their application's deployment
configurations and roll them out completely automated. Furthermore, when
applied correctly, the concepts save time that would otherwise be spent on
manual labour.
