%! TEX root = ../../main.tex

\section{Conclusion \& Outlook}%
\label{sec:Conclusion}

This thesis started with the goal of solving two problems commonly associated
with a continuously deployed microservice architecture. To achieve this goal,
the \ac{DSR} research framework was used to guarantee an outcome that would be
tested in practise as well as theoretically applicable to similar problems. The
two research questions that were discussed in this thesis were:

\begin{itemize}
  \item How should a continuously deployed microservice be versioned?
  \item How should microservice manifests be managed and continuously deployed?
\end{itemize}

The actual task of deploying a microservice architecture was considered to be
trivial. Therefore, it was only outlined as part of the background chapter and
not further examined.

The first research question was answered in chapter~\ref{sec:Versioning}. A
classification model for microservices was introduced which distinguishes
between \textit{consuming} and \textit{producing} microservices. Based on this
classification, the terms \textit{manual} and \textit{continuous} versioning
were coined. If applicable, microservices should be assigned a version number
using a continuous versioning scheme. One example for a continuous versioning
scheme is CalVer. CalVer defines a set of rules for version numbers that are
based on calender dates. Thus, they can be assigned automatically without human
intervention. In cases, where the version number of a microservice has to
indicate a break in backward-compatibility, a manual versioning scheme like
SemVer should be used. If SemVer is technically required, it can also be
customized to be partially continuous by e.g.\ appending a rolling build
version at the end of the version string. In addition, a guideline was
developed for developers which helps them to identify the best versioning
scheme for their current project.

The second research question was resolved in
chapter~\ref{sec:Handling_Kubernetes_Manifests}. First, a naive deployment
approach was outlined. On the basis of this approach, a new one was developed
which solves a number of issues related to deployment configuration management.
The newly developed \ac{CIST} deployment concept is based on Helm, a package
manager for Kubernetes. With \ac{CIST}, a microservice architecture's
deployment configuration state is defined a central location, a Helm chart
repository. The architecture is divided into \textit{component charts} and
\textit{master charts}. A component chart defines how a single microservice has
to be deployed. A master chart glues all microservices of an microservice
architecture together and defines the architecture's underlying infrastructure.
From the chart repository, a complete microservice architecture including all
its components can be deployed without any additional manual interaction. The
only component that has to be deployed is the master chart. Everything else is
put together by Helm. \ac{CIST} also defines the procedure that produces said
central state. This procedure can be applied to any microservice architecture
which developers aim to simplify the project's deployment process. In addition
to an easier deployment process, Helm offers additional advantages like a
simple rollback strategy.

The outcome of both research questions can be easily transferred to similar
projects. However, the practical implementations of both solutions only
represent a reference and no completed solutions. Thus, only one possible
improvement for the implementation of \ac{CIST} is presented. When generating a
component chart and its corresponding master chart, a SemVer versioning number
is updated. In the proposed implementation, the version number is always set to
\texttt{1.0.0+\textbf{build-number}}. If the manual part of the version number
was set to anything other than \texttt{1.0.0}, these manually set values would
be overwritten. To solve this problem, the old version number could be read
from the chart's configuration file. The build number could then be appended to
the dynamically read manual part of the SemVer version number.

Nonetheless, all theoretical findings of this thesis are applicable without
constraints in a project that uses a microservice architecture. Finally, this
thesis' findings support developers whose applications are publicly accessible
on the internet to continuously serve their applications to the demand of
billions of possible users.
